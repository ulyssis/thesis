\chapter{Abstract}
Due to the proprietary spectrum allocation paradigm and the boom of wireless communications, spectrum scarcity has become an increasingly pressing problem. 
Cognitive Radio is a promising technology to provide high bandwidth via dynamic spectrum access, where unlicensed users  are allowed to utilize licensed spectrum, as long as their operation doesn't cause harmful interferences, or no primary users are sensed with respect to a certain probability.
This spectrum usage paradigm prompts new challenges to both research and application.
One challenge is the spectrum allocation problem needs reformulation as primary users should be considered in the problem formulation.
The second issue is how to improve the accuracy of sensing result by applying robust cooperative sensing.
The robust clustering maintains the clusters for longer time under primary users' unexpected activities, so as to gains more benefits from cooperative sensing.
%that when clustering is needed to facilitate local cooperation, as the amount of available spectrum on unlicensed users is different, it is necessary to form nearby unlicensed users which have similar spectrum together.
%thus choosing a suitable neighbour to communicate with wider spectrum band 
Routing on cognitive radio network needs to consider the influence from primary users besides the conventional concerns in ad hoc network.

As available spectrum or resources for cognitive radio users is dependant on their locations, cognitive radio users are suitable to make decisions autonomously, with or without utilizing the knowledge from neighbourhood or centralized entity.
This thesis solves a series of problems with distributed solutions.
As a suitable theoretical tool, game theory is used to analyse certain problems and help to derive and verify solutions.
%Distributed solution is more suitable for wireless networks, because it doesn't require global information across the network, and individual unlicensed user is either agile to react to network dynamics, or only needs local information.

In this dissertation, we propose distributed solutions for the problems residing from physical layer up to network layer in CRN. 
%In this dissertation, we solve three problems residing from physical layer up to network layer in CRN with distributed solutions.
%Based on game models, we design distributed algorithms for CR users to self-organize and utilize the available resources, which doesn't lead to global instability or performance deterioration.
%
We solve the power and spectrum allocation problem in IEEE 802.22 networks.
After deciding the maximal transmission power on each secondary cellular base station, we formulate the distributed spectrum allocation problem in TV white space scenario into a canonical congestion game, then propose distributed algorithm corresponding to the behaviour of player in the game.
Power allocation is then conducted on the channel decided before.
%
%cognitive radio users form clusters with neighbours, where they share certain common channels, in order to do cooperative sensing.
On MAC layer, a distributed clustering scheme for CRN is proposed and the formed clusters are much less vulnerable against primary users' activity, besides, the scheme is able to produce desired cluster sizes.
The process of finalizing cluster membership is innovatively formulated into a congestion game, and light-weighted distributed scheme is accordingly derived.
The proposed clustering scheme can be easily applied to other scenarios, where certain similarity exists in local area.
%
On network layer, we propose lighted weighted geographic routing scheme for CRN. 
Spectrum aware virtual coordinate is proposed, thus light weighted geographic routing can be applied to decide the next hop.
%
Finally, using thorough simulations and numerical results, we investigate various aspects of the proposed algorithms and analyse their performance.


%The \Gls{latex} typesetting markup language is specially suitable 
%for documents that include \gls{maths}.