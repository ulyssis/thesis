\chapter{Abstract}
Due to the proprietary spectrum allocation paradigm and the boom of wireless communications, spectrum scarcity has become an increasingly pressing problem. 
Cognitive Radio is a promising technology to provide high bandwidth via dynamic spectrum access, where unlicensed users  are allowed to utilize licensed spectrum, as long as their operation doesn't cause harmful interferences, or no primary users are sensed with respect to a certain probability.
This spectrum usage paradigm prompts new challenges for spectrum utilization in many aspects, such as deciding the spectrum which is available and suitable for communication, deciding the maximum transmission power to avoid harmful interference at primary users, delivering service, etc.

%One challenge is the spectrum allocation problem needs reformulation as primary users should be considered in the problem formulation.
%The second issue is how to improve the accuracy of sensing result by applying robust cooperative sensing.
%The robust clustering maintains the clusters for longer time under primary users' unexpected activities, so as to gains more benefits from cooperative sensing.
%%that when clustering is needed to facilitate local cooperation, as the amount of available spectrum on unlicensed users is different, it is necessary to form nearby unlicensed users which have similar spectrum together.
%%thus choosing a suitable neighbour to communicate with wider spectrum band 
%Routing on cognitive radio network needs to consider the influence from primary users besides the conventional concerns in ad hoc network.

Due to the propagation path attenuation of spectrum, the available spectrum or resources at cognitive radio users are dependant on their locations, which are usually different from place to place.
Furthermore, the available spectrum varies due to primary users' activities.
Thus, a secondary user is suitable to make decisions autonomously and only based on the local information, as it is easier for a secondary user to know the updated spectrum availability on it.
%, with or without utilizing the knowledge from neighbourhood or centralized entity.
In this thesis, we design distributed algorithms to solve the problems imposed on cognitive radio networks (CRNs), and show the advantages of distributed solutions over centralized schemes in CRNs.
In the process of algorithm design, we employ game theory to guarantee the dynamics within CRN to converge into Nash equilibrium.
%Distributed solution is more suitable for wireless networks, because it doesn't require global information across the network, and individual unlicensed user is either agile to react to network dynamics, or only needs local information.

The concrete problems solved in this thesis reside from physical layer up to network layer in CRNs.
%Based on game models, we design distributed algorithms for CR users to self-organize and utilize the available resources, which doesn't lead to global instability or performance deterioration.
%
On physical layer, We propose method to let secondary users choose channel and power in a network, whose architecture is compatible with the regulation on the TV database access.
%We solve the power and spectrum allocation problem in IEEE 802.22 networks.
After deciding the maximal transmission power on each secondary cellular base station, we formulate the distributed spectrum allocation problem in TV white space scenario into a canonical congestion game, then propose distributed algorithm enlightened by the player behaviours in the game.
Power allocation is then conducted on the channel decided before.
%
%cognitive radio users form clusters with neighbours, where they share certain common channels, in order to do cooperative sensing.
On MAC layer, a distributed clustering scheme for cognitive radio ad hoc network (CRAHN) is proposed.
CRAHN adopts cluster structure to obtain more accurate sensing result, or to facilitate network management.
In order to resistant the compulsory evacuation of channel due to primary users' random appearance, the formed clusters should poses more common channels within them, and should be able to control their clusters sizes.
The negotiation among neighbouring secondary users to form clusters with above characteristics is formulated into a congestion game, then a lightweight distributed scheme is derived.
%The proposed clustering scheme can be easily applied to other scenarios, where certain similarity exists in local area.
%
On network layer, we propose a lightweight geographic routing scheme which works with spectrum aware virtual coordinate for CRN. 
The spectrum aware virtual coordinate integrates the spectrum availability between any pair of neighbouring secondary users, in particular, the distance between two secondary users is longer when the available spectrum is more scare in between them.
%
Except for the last problem, we also propose centralized schemes to give bounds on the performance.
We compare distributed and centralized schemes in the aspects of singling overhead and time complexity, and analysis the strength and weakness of the two in different CRN scenarios.

%solving these problems, w
%Finally, using thorough simulations and numerical results, we investigate various aspects of the proposed algorithms and analyse their performance.


%The \Gls{latex} typesetting markup language is specially suitable 
%for documents that include \gls{maths}.