\chapter{Introduction}
Wireless networks utilize radio waves instead of wires to carry information, hence seamless coverage and mobility are achievable in wireless connections.
Wireless networks have experienced unprecedented growth in the past few decades and will evolve in the future.
Due to the propagation characteristics of radio waves, only a small part of radio spectrum which spans from 3 kHz to 300 GHz
is suitable for commercial application.
In most countries, the radio spectrum is divided into different bands (also referred as channels), some of which are licensed to different types of usages and users by governments through auction, and the governments get revenue from it~\cite{Spectrum_Management07}.
There are many existing wireless applications assigned with licensed spectrum.
For instance, the second-generation (2G) wireless cellular network GSM (global system for mobile communications) in Europe works with GSM-900 band (from 890 MHz to 960 MHz) and GSM-1800 (1710 MHz to 1880 MHz), and the third-generation (3G) wireless cellular network works from 1.8 GHz to 2.4 GHz~\cite{wireless_communicatioins2001}.
The assignment of frequency band rules out the unlicensed users to use the spectrum, so as to avoid interferences caused on the licensed users, and protect licensed users' commercial benefits.

The proliferation of wireless network causes significant shortage of spectrum under current spectrum allocation paradigm.
Open spectrum access~\cite{osa_Noam_1995} is proposed in 1995 to cope with the scarcity of spectrum at certain places during certain time, which advocates a new spectrum usage paradigm where spectrum use does not require any license.
On the other hand, there exists a large number of spectrum bands which have considerable dormant time intervals, and the spectrum is not fully utilized~\cite{Akyildiz06survey}.


XXXXX Figure of spectrum usage XXXXXX


\section{Cognitive Radio}
The dilemma that spectrum scarcity coexists with spectrum underutilization promotes cognitive radio (CR) as a promising technology to make full use of spectrum and accordingly solve the spectrum shortage problem.
Cognitive radio is a device which is able to sense, detect, and monitor the surrounding radio frequency environment, then based on the assessment along with the location information and certain particular operating rule, the cognitive radio tunes its radio operating parameters (e.g. center frequency, bandwidth and transmit power) on the fly so that to avoid interfering licensed users.
The definition of cognitive radio evolves with the development of radio technology and regulations.
We choose two representative definitions to give a formal description of cognitive radio.
Cognitive radio is firstly proposed by Mitola III who defines the concept of CR in his dissertation~\cite{2000mitola_cognitive_radio} as follows:
\blockquote{The point in which wireless personal digital assistants (PDAs) and the related networks are sufficiently computationally intelligent about radio resources and related computer-to-computer communications to detect user communications needs as a function of use context, and to provide radio resources and wireless services most appropriate to those needs.
}

FCC (Federal Communications Commission in U.S.) describes CR~\cite{FCC_03-322} as,
\blockquote{
a radio that can change its transmitter parameters based on interaction with the environment in which it operates. $\ldots$
This interaction may involve active negotiations with other spectrum users and/or passive sensing and decision making (smart radio) within the radio. The majority of CRs will probably be SDRs~\footnote{software defined radio is a radio communication system which is able to receive any modulation across a large frequency spectrum, and transmit on desired spectrum band.}, but a CR does not necessarily use software, nor does it need to be field programmable.
}

The wireless network which is composed with cognitive radio users is cognitive radio network, the acronym is CRN.

\subsection{Spectrum Access Etiquette in Cognitive Radio Network}

%CR users utilize the unused licensed spectrum opportunistically and meanwhile avoid interfering licensed users.
%This new spectrum usage paradigm imposes great challenge to the CR users, including how to detect the licensed users and afterwards decide on the suitable spectrum and on which interact with other CR users.

%Before the emerging of cognitive radio technology, spectrum is allocated in a fixed way, \ie certain chunk of spectrum is exclusively licensed to an certain entity by administration body.
%The equipments which are given the license to utilize the spectrum is called licensed users, while the others which called unlicensed users are not allowed to work on the that part of spectrum.
%This paradigm rigidly protects the benefit of licensed users, but it results in underutilization of spectrum.
%Due to the drastic increase of mobile data applications there is an urgent demand on spectrum resources. 
%Traditionally, frequency bands have been assigned exclusively to licensed bodies which can occupy the spectrum whenever there is information to be transmitted. 

%This static spectrum utilization has been proven to be suitable for many systems and applications. 
%However, as mobile networks have proliferated drastically over the last thirty years, this exclusive assignment has created a significant shortage of spectrum today. 

Licensed users access their allocated spectrum band whenever there is information to be transmitted, in contrary, CR users are only allowed to access licensed spectrum after validating the channel is idle or the primary user is not to be affected if they operate on the licensed spectrum.
In this thesis, licensed users are referred as primary users, and CR users are denoted as secondary users.

The assessment of spectrum is a bone of contention for primary/secondary users and administration bodies.
Spectrum sensing on secondary users is one common method to validate spectrum availability especially in research domain~\cite{crnsensing_09}.
Secondary users should monitor the spectrum of interest actively and autonomously to detect primary users' appearance.
Primary users can be detected by judging the primary users' signal power, spectral correlation or beacons.
Spectrum sensing requires sophisticated technologies when primary users' signal is weak.
Primary detection can be improved by learning technologies or cooperation among multiple secondary users~\cite{coorperativeSensing_Akyildiz11}.
Another way to protect the primary users from being affected by secondary users' operation relies on location identification and certain operating rule set.
Based on the global information of primary users' location and terrain information, centralized controller regulates that the secondary users at certain locations are restricted to operate on a few certain spectrum bands and with limited transmit power~\cite{whitefi09}.
Spectrum administration bodies FCC of U.S.~\cite{FCC_2010_sedond_memorandumm} and Electronic Communications Committee (ECC) in Europe~\cite{ecc159} adopt the combination of spectrum sensing and location based method.
Secondary users' operation is restricted on certain spectrum bands and transmit power should be below certain threshold according  to their locations, and spectrum sensing ability is also required.



%In contrast, CR users (forming cognitive radio networks, abbreviated as CRN) 
%This refers to the process of sensing a particular channel and verifying (with a previously specified probability of error) that it is not used by a primary user currently  [cite spectrum sensing].
%This form of spectrum sharing is also referred to as opportunistic spectrum access~\cite{Akyildiz06survey}.

%Such co-existence with primary users imposes great challenges for the CR users.
%On the first hand, CR users should be able to sense the channel to avoid interfering primary users.
%it is fairly easy to see that the sensing ability of secondary users plays an important role in the harmonious co-existence of primary and secondary users~\cite{09spectrumSensing_survey}.


\subsection{Spectrum Decision}

After assessing RF environment or geographic location, secondary users adjust their operational parameters such as frequency, modulation schemes and transmit power, in order to support QoS aware communications, this process is referred as spectrum decision and spectrum sharing~\cite{08crn_survey}.
In this thesis, spectrum decision and sharing consist the major issue discussed in this thesis.

Some work in research community models the spectrum availability with stochastic or statistic model, which is helpful when deciding which channel to use.
\cite{Discrete-Time_Spectrum_Occupancy_Model_DySPAN_2011} proposes discrete Markov chain and adjusted duty circle models to describe the availability of licensed spectrum for GSM on 900/1800 MHz.
\cite{Wellens200910} models the channel holding time with geometric and log-normal distributions.
Statistics of previous sensing results is used to predict spectrum state in the future in \cite{spectrum-discovery-tmc08}.
Such models provide more complete information on the availability of the licensed channels.

The available licensed spectrum which spans a wide frequency band exhibits different characteristics~\cite{spectrum_decision_TMC11}.
Based on the requirements of interested communication, CR users need to identify the characteristics of the spectrum, which include channel quality (channel capacity, error rate, path loss, etc.)~\cite{spectrum_decision_TMC11}, channel switching delay~\cite{channel_switch_delay11}, and channel holding time, \ie the expected time duration that the primary users don't occupy the channel before any one occupies again.


\section{Cognitive Radio Network}
In this section, we introduce two types of cognitive radio networks, and the problems we tackle in this thesis reside in these networks.

\subsection{Cognitive Radio Ad hoc Network}
An ad hoc network is a decentralized paradigm of wireless network, which consists of a collection of autonomous mobile users which communicate over wireless links.
Efficient distributed algorithms are needed to determine network organization, link scheduling, and routing.

Cognitive radio ad hoc network is an ad hoc network composed with cognitive radio users.
Similar with ad hoc network, cognitive ad hoc radio network can also be represented as an undirected graph $G$.
Cognitive radio users constitute the vertices, the edge between two vertices is decided not only by the distance, propagation and attenuation properties between the two vertices, but also the spectrum availability on both vertices, \ie, when they can decode the received signal from each other correctly, and there is common channel available between them on which communication is conducted, then an unidirectional edge is available on graph $G$.
As to ad hoc network, the graph is constant when users are static.
As to cognitive radio ad hoc network, due to primary users' activity, an edge between two vertices is decided by the fact that whether the two vertices can simultaneously access the same licensed channel.
Hence, the corresponding graph is dynamic under primary users' operation, which imposes extra difficulties on network organization, routing and many other network functionalities.

\subsection{IEEE 802.22 standards}
Centralized spectrum decision is adopted in IEEE 802.22~\cite{802.22} standard for Wireless Regional Area Networks (WRAN).
IEEE 802.22 defines a cellular network paradigm for secondary equipments working on unused TV channels.
Centralized database notifies the secondary users the available spectrum at their places, and is possible to decide transmission parameters for them, \ie spectrum to be used, or transmission power.
Note this database takes the functionality of spectrum sensing in addition to spectrum decision and sharing.
The feasibility of this centralised paradigm is largely due to the characteristic of TV channel, as TV channel usage follows a slow and scheduled pattern. 

Centralized spectrum decision doesn't work well outside the TV channel scenario.
In certain scenarios, primary users are active and the channel holding time is short, thus the channel availability changes frequently.
Furthermore, it is hard for CR users to obtain a full and up to date picture of the spectrum availability in the whole network.
As a result, spectrum decision and sharing need reconsideration quite often, which causes a large number of control messages for network organization.
As contrary, distributed schemes adapts to the varying environment quite well~\cite{Selforganization_CRN_13}.
Distributed schemes exploit local observation, and require much less control overhead compared with centralized optimization.

The forthcoming vehicular to vehicular communication requires

%Based on the type of primary system which CRN underlays to co-exist, the availability of licensed spectrum exhibits variations on temporal and spatial aspects.



%However, the autonomous operation of secondary users always comes with a residual risk that primary systems are not detected despite their reappearance. Hence, other approaches to primary detection have been proposed relying on geolocation and database (DB), where secondary users are told by a centralized database about the spectrum availability with a time dimension. In this concept, the transmission power can also be controlled by the database. After obtaining the available spectrum to use, the secondary users need to decide which chunk of spectrum to use so as to satisfy the QoS requirements for the service it conveys. In order to achieve good performance, secondary users should be aware of the activity pattern of PUs, so that they can pro-actively plan the spectrum usage. Finally, SUs need to share the spectrum with other SUs, thus interference mitigation is an important issue to be considered.

%Whenever primary users are detected, the secondary users have to either jump to other spectrum, or turn down its transmission power to avoid interfering the primary users.

\section{Channel Allocation in TV White Space}
\subsection{Wireless Channel Allocation}
xxxx what is channel allocation? xxxx

Channel allocation facilitates CRN to improve throughput~\cite{channelAllocation_throughput_12wcnc}, or cooperatively relay~\cite{channelAllocation_relay_2010ICASSP} and so on.
This thesis emphasises on co-channel interference mitigation with distributed channel allocation . 

Mitigating co-channel interference via channel allocation has been attracting plenty of research efforts in the past ten years, from multiple channel mesh network~\cite{Hyacinth}, Ad hoc network~\cite{Babadi08, Ko_DistributedCA} up to cognitive radio network~\cite{SA_CA_TVWS_2012crowncom,qlearning_huang}. 
Channel allocation problem is converted into colouring problem thus is NP hard~\cite{Hyacinth}, thus centralized optimization fails to produce
Authors of~\cite{Babadi08, Ko_DistributedCA} propose heuristic algorithms utilizing best response based on the welfare on itself to assign channels among users.
Simulated annealing is applied to mitigate co-channel interferences in~\cite{SA_CA_TVWS_2012crowncom}.
For the same purpose, No-regret learning~\cite{qlearning_huang, hart00correlatedeq} is exploit to optimize the choice on channel.

In this thesis we cope with a special channel allocation problem where symmetric interaction doesn't exist, \ie transmission power is identical among CR users, or the propagation path loss is not symmetric. 
The asymmetry disables the heuristic distributed schemes provided in~\cite{Babadi08, Ko_DistributedCA}, and makes channel allocation problem not to fit into the congestion game model proposed in~\cite{allerton08_liu} which is the first paper to discuss channel allocation from the respective of game theory.
We innovatively formulate this problem in to a canonical congestion game by utilizing the centralized database in TV white space scenario, and derive efficient distributed channel selection strategy.
%and apply it on different cognitive radio networks.
%rethink channel allocation problem from the perspective of game theory, particularly,

\subsection{Utilize TV White Space}
Unused TV spectrum is termed as TV White space by the Federal Communications Commission (FCC)~\cite{FCC_2010_sedond_memorandumm}, which is licensed to incumbent users such like digital TV, analog TV, and wireless microphone.
As to unlicensed users, detecting incumbent users is challenging because the FCC requires the unlicensed users should be able to detect the presence of signals from TV stations or wireless microphone at a received power level of -114 dBm~\cite{Technical_Challenges_TVwhit}. 
Thus FCC doesn't require the sensing ability on unlicensed users, but regulates the secondary usage of TV white space in a prudent manner, including the spectrum bands permitted to use based on their location, the transmission power, the distance away from TV service area and so on.
Every unlicensed user should register its type and geographic location to one TV database which decides which channels can be used at its places, then unlicensed user accesses the TV database to retrieve the information about available spectrum.
%This is very applicable as TV spectrum usage changes slowly, and the spectrum usage by TV stations is scheduled.
%the geographic location approach together with central database becomes more appealing in TW White Space (TVWS) utilization scenario. 
% FCC regulates portable secondary users to operate from channel 21 (512 MHz) to 51 (698 MHz), with the exception of channel 37. As to fixed secondary usage, the allowed spectrum band is from TV channel 2 (54 MHz) to TV channel 51, with TV channels 3, 4 and 37 being prohibited. Thus, the available TV spectrum is about 600 MHz wide. Compared to conventional unlicensed ISM bands in the 2.4 GHz and 5 GHz band, all together TVWS has more to offer.
Some prototype applications which only rely on the TV database are proposed in cellular network~\cite{tvwhite_lte2011, multicell_geo_dyspan11} and WiFi-like network~\cite{whitefi09} based on the FCC regulation.
  
Standardization activities are also ongoing on TVWS utilization, including 802.22~\cite{802.22} for Wireless Regional Area Networks (WRAN), IEEE 802.11af~\cite{802.11af} for WLAN, IEEE 802.15.4m~\cite{802.15.4m} for 802.15.4 wireless networks in TVWS and 802.19.1~\cite{802.19} for coexistence methods among local and Metropolitan Area Networks (MAN).
IEEE 802.22 largely complies FCC regulations on the utilization of TVWS.
System consists of base stations and customer premises equipments (or terminals for short), where each terminal is served by one base station. 
Recent standard published in Nov. 2010 suggests both sensing ability as well as database look-up to avoid affecting primary systems.
As to utilization of available TVWS, IEEE 802.22 proposes centralized channel allocation in database.
When two or more base stations co-exist on the same channel, TDMA like mechanism for WBSes is adopted.
%\cite{HoangPowerChannel2010} proposes a distributed solution for power control and channel assignment in both down-link and up-link communication in a WRAN, but the investigated secondary network is composed with only one base station and multiple terminals.

Scientific research on utilization of TVWS goes on in parallel with the regulatory agency.
Spectrum sharing in TVWS is formulated as optimization problem, where the guarantee that TV receivers should not be affected by the cumulative interferences form TVBD is one constraint, and the signal interference (noise) ratio becomes the other.
The objective can be maximizing TVBD's downlink transmission power~\cite{multipleIntf_pimrc11}, uplink transmission power~\cite{uplink_power_tvws13}, or best geographic distribution of TVBDs~\cite{withinTVcoverage_PIMRC13}.

\cite{game_CA_association_ICDCS12,SA_CA_TVWS_2012crowncom, 802.22co-existence09, 802.22game_08globecom,self-coexistenceWRAN2010infocom} emphasise on interference mitigation among TVBDs via spectrum allocation.
Vehicular networks operating with TVWS assisted by TV database and cooperative sensing is discussed in~\cite{tvws_vtc13}.
Work~\cite{increaseTVWS12} steps further from the database paradigm and makes efforts to utilize the 'grey space', where TVDB is allowed to operate even within the TV service area.

This thesis addresses following two problems,
\begin{itemize}
\item Decide the maximal downlink transmission power.
Both FCC regulation and 802.22 standard try to make TVBD transparent to incumbent users, but as long as TV system is not affected, i.e. certain quality of service is fulfilled, the strict restriction on unlicensed users can be relaxed so that more TVWS can be provided~\cite{multipleIntf_pimrc11}. 
Abiding by the operation paradigm using data base, we investigate the maximal downlink transmission power for TVBDs by solving optimization problem where the cumulative interference on TV receivers is under a threshold.

\item Distributed spectrum allocation scheme for TVBDs.
According to 802.22 regulation, spectrum allocation is done centrally in TV database, this is not realistic when TVBDs belong to different economic interest groups, thus a distributed solution is needed.
We propose efficient distributed scheme to allocate the TV channels in order to improve the quality of service of TVBDs.
The major difference between our scheme and other spectrum allocation lies in that the downlink transmission power on different channel is different.
We formulate this problem into a canonical congestion game, and derive the distributed algorithm from the best response behaviour of the player in the game. 
\end{itemize}



\section{Clustering in Cognitive Radio Network}
Clustering is an important paving stone for the practical utilization of the unused portions of the licensed spectrum.
Clustering secondary users based on geographical proximity and other relevant properties together produces following benefits.
Firstly, it is more efficient to solve common control channel (CCC) problem with cluster structure.
Dedicated CCC which is allocated to all nodes for the purpose of control information exchange is regarded to be under utilization.
Whereas, cluster based approaches group CR nodes into clusters based on their similarity of available unlicensed channels, so that the common channels within each cluster are used to carry the control messages~\cite{Lazos09}.
%whereas communication rendezvous, \ie the process to establish control channel between two CR users before they can communicate is proposed to be a economic solution.
%Within one cluster which is composed with CR nodes with similar available unlicensed channels, communication rendezvous can be accomplished within in shorter time~\cite{CommunicationRendezvous_ToN13}.
Secondly, cluster structure facilitates cooperative sensing and increases the sensing reliability~\cite{Sun07_clustering_spectrum_secsing}.
Thirdly, cluster structure supports coordinated channel switching and simplifies routing in ad-hoc cognitive radio networks~\cite{cluster_routing_2013ICC}.

A lot of research effort has been made on distributed clustering in CRN, these work target different aspects.
In~\cite{Zhao07, Affinity_clustering_09icccn}, the channel available to the largest set of one-hop neighbors is selected as common channel which yields a partition of the CRN into clusters.
Schemes~\cite{LIU_TMC11_2, cluster_EW10} pursue high numbers of common channels within clusters, so that cluster common control channel is less likely to disappear or encounter traffic congestion.
Work~\cite{Consensus_based_clustering12} improves spectrum sensing ability by grouping the CR users
with potentially best detection performance into the same cluster.
Clustering scheme~\cite{clustering_globecom11} obtains the best cluster size which minimizes power consumption caused by communication within and among clusters.


There are three aspects need consideration when design a new clustering scheme.
\begin{itemize}
\item Abundance of control channels within cluster should be achieved.
A large number of control channels within cluster means high robustness.
When the current control channel gets occupied by primary user, cluster members can migrate to a new one and the cluster is maintained.
Besides, more control channels makes multiple concurrent transmission within cluster possible.
In this thesis, a distributed clustering algorithm which is especially designed to support robustness under active primary users is proposed.
Related works~\cite{Zhao07,Affinity_clustering_09icccn,Consensus_based_clustering12,clustering_globecom11} fail to pay attention to this aspect.

\item New scheme should be light weighted so that re-clustering can be quickly conducted when previous cluster is destroyed by primary user's activity.
When all the common channels are occupied by primary users, cluster head selection and following procedure is conducted by the cluster members autonomously.
\cite{LIU_TMC11_2} targets large number of control channels within cluster, but it intriguers high complexity.


\item Efficient channel allocation scheme within and among clusters is needed, so that communication rendezvous between two clusters is quick. 
Communication rendezvous means the process to establish control channel between two clusters before they can communicate .
\cite{LIU_TMC11_2} proposes channel allocation in round robin manner, but it causes long time on communication rendezvous.
\end{itemize}

These requirements will be fulfilled by the scheme proposed in this thesis.




%\section{Spectrum Aware Virtual Coordinates in CRN}
\section{Geographic Routing in CRN}
We propose a routing paradigm in CRN.
Geographic routing is applied in the CRN network which is assigned with spectrum aware virtual coordinates.
The dynamic availability of spectrum leads to prevalent topology changes, which makes spectrum aware routing difficult but essential.
Routing schemes are proposed in~\cite{Abbagnale_Gymkhana10, caodv-10wd, segment-crowncom08} for CRN where primary users change their operating parameters infrequently.
Highly dynamic primary users impose great challenge on routing, as is discussed in~\cite{Routing-crn-INFOCOM11}, where the statistics of primary users' activity is utilized in routing decision.
A class of packet forwarding strategies for dynamic spectrum CRN is proposed in~\cite{routing-crn-icc11, routing-crn-jsac12}.
Whenever a secondary user needs to forward a packet, it chooses channel and hop jointly based on channel's statistical characteristics observed beforehand.
Forwarding decision is made for each single packet, which requires complex computations, large amount of control overhead, and customized media access control mechanisms.
The solution provided by Chowdhury et al.\cite{search_geo_routing_chowdhury} improves geographic routing in multiple channel CRN by introducing circumventing mechanism, \ie when the next hop chosen based on geographic routing metric (\eg Euclidean distance) is affected by primary user, the routing packet chooses a neighbour of that node free from primary user's affection so as to avoid the primary user affected area.
Such routing is conducted on all channels, afterwards a path merge process is undertaken and one path with alternating channel is finally formed with consideration of end to end delay.

As the decision of the next hop is largely decided by the channel availability on the time point of decision, the node chosen as next hop may not be able to work after a short while due to primary user's reappearance.
Thus, this scheme works well when the primary user's activity is infrequent, but when it goes tense, the frequent invalidity of nodes due to lack of available spectrum seriously deteriorates routing performance.

In this paper we propose SAViC, spectrum aware virtual coordinates for secondary users in multi-channel multi-hop CRN where secondary users are source limited.
Virtual coordinate is independent of real geographic position, but decided by certain properties of the media among nodes, for instance, link quality or hop numbers~\cite{gpsfree05infocom}.
The proposed virtual coordinate depicts the availability of licensed spectrum influenced by primary users, on top of which geographic routing decides the next hop with Euclidean distance metric, and unconsciously detours the primary affecting area, or cuts through the area with better access opportunity.
This routing paradigm imposes little computation and communication cost on secondary users after assigning virtual coordinate, besides, it doesn't need real geographic location which is employed in ~\cite{search_geo_routing_chowdhury, routing-crn-jsac12}.

This scheme is composed with two steps,
\begin{itemize}
\item Design virtual coordinates so that virtual coordinates of any two secondary users reflect both geographic distance and opportunistic spectrum availability between them.
We design them based on statistics of primary user’s ON/OFF states which are obtained from local spectrum sensing.

\item After deciding on the next hop, we adopt a lightweight heuristic method to decide which channel to transmit packet when multiple licensed channels exist in the network.


\end{itemize}

To summarize, as the Euclidean distance between two secondary users based on spectrum aware virtual coordinate reflects the availability of unlicensed channel in between from the angel of historical statistics, virtual coordinate contributes a large part to deciding on the on the next hop. 



