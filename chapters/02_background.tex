\chapter{BACKGROUND}

%Some mathematical techniques which are used to solve the problems in thesis are introduced in this chapter.

In the past few years, game theory was extensively applied to problems in communication and networking~\cite{Neel06analysisand, Wang_gtc_crn_survey_2010}.
Game theory is a mathematical tool for studying, modelling and analysing the interaction of rational decision makers which have conflicting objectives.
When the network entities are seen as rational, game theory can be used to predict and guide the behaviour of them, and predict the outcome of the system with respect to certain metrics.
In contrary to game theory which takes care of the behaviour of all decision makers, optimization problem is only conducted on a single decision maker, which attempts to achieve the best welfare for that decision maker.
In the following, we introduce some basics of game theory and optimization.





\section{Algorithmic Game Theory}


In cognitive radio networks, game theoretic models are used to better understand resource allocation, topology control, routing, security and many other issues in wireless communication systems.
%In recent years, game theory attracts people attention into apply it in communication systems.
According to~\cite{game_for_communication_01}, there are several reasons to apply game theory in communication systems.
\begin{itemize}
\item Communication equipments are supposed to be manufactured and operate based on standards to fulfil certain functions, but selfish behaviour may appear on certain individual equipments to achieve advantages over their peers.

For instance, Wi-Fi equipments are manufactured complying the IEEE 802.11 standards.
But it is possible that certain manufacturers or the personal who uses the Wi-Fi system (we use station in the following) manipulate certain parameters to achieve performance advantage over other stations in the network.
When all the stations in one network are supposed to run distributed coordination function (DCF), \ie the station must wait a random period of time (contention window) before accessing the media when it senses the media to be busy, a certain selfish station may not choose to wait and is keen on sensing media.
The selfish behaviour causes more collisions for other stations, and possibly results in poor performance in the network.
%In this case, the access point finds more packets coming from that selfish user, but is unaware of its selfishness and has no means to adjust or punish this selfish user.
In this case, game theory facilitates the network operator to issue rules to make the selfish behaviour unprofitable, or it helps to analyse how much the impact caused by the selfish stations.
% can tell the access point or network operator the impact caused by the selfish systems, and possibly 

\item Algorithms can be retrieved from the analysis of a problem under the game theoretical framework.
In the same example of media access in IEEE 802.11 DCF mentioned in previous item, if stations are allowed to modify the length of contention window, each station will adjust its contention window to obtain the best performance.
\cite{contentiongame_07} shows as long as each station greedily updates its CW to maximize certain utility, after certain time it will stop to do so as the current CW is the best as to the performance.
In the case, the best response becomes a algorithm for Wi-Fi systems.
%They interact with each other and update their contention windows, when they reach Nash Equilibrium state, no system has motivation to change its contention window any more.


%For example, every internet equipment embeds TCP/IP protocol based a series of Request for Comments (RFC) documents, but it is possible that some manufacturers or people who use them may modify or reconfigure the , so that their equipments work selfishly to seek performance advantage over other equipments.
\item Game theory assumes the research objectives to be rational.
In communication system, a device is programmed to maximize or minimize the expected utility, which is perfectly rational.

\item Outcome is robust\cite{Han:2008:RAW:1457343}.
As to optimization, when the information is not accurate or adequate enough, or the optimization itself is difficult to solve, the optimized results can be far from global optimality.
Game theory based on the local information, which is easy to obtain, usually leads to Nash equilibrium, which is usually sub optimal.


\item 
Only local information is needed, thus game theory is naturally suitable for designing distributed solutions for cognitive radio networks.
Distributed solutions doesn't rely on centralized controller, and each user in the network adopts certain action as response to the other users or environment, this falls in the scope of game theory.

\item Combinatorial nature.\cite{Han:2008:RAW:1457343}
Many problems in wireless communication involve integer variables, \ie channel assignment, selection of modulation levels or channel coding.
It is always challenging to solve combinatorial optimization problems, whereas game theory is natural to describe it in a discrete form.




\end{itemize}



 

%\todo[inline]{expand: Introduction of game theory...}
\subsection{Basics of Game}
%Game theory is established by Nash in xxxx, and is applied in network communication since xxx
In this section, we give a brief introduction of game theory along with particularly congestion game which is used in our thesis.
We briefly list the important concepts of algorithm game theory here, many of which are referred from ~\cite{agt_book}.


\subsubsection*{Strategic Game}

A strategic game can be represented as a tuple $\Gamma = (\mathcal{N}, (\sum_i)_{i \in \mathcal{N}}, (u_i)_{i\in \mathcal{N}})$, where $\mathcal{N}$ is a finite set of players, $\sum_i$ is player $i$' finite strategies.
Player $i$ selects one strategy $s_i\in \sum_i$ at one time to play the game.
$u_i:\sqcap_{i\in \mathcal{N}}\sum_i\rightarrow \mathbb{R} $ is the utility function of player $i$.

Some relevant common used items are introduced.
$\sum=\sqcap_{i\in \mathcal{N}}\sum_i$ is the set of states of the game, which denotes all the possible ways that players pick strategies.
$\sum_{-i}=\sqcap_{i\in \mathcal{N}\setminus \{i\}}\sum_i$ is the set of states of all the other players except for player $i$.
$\sigma=(s_i,\cdots,s_n)$ is called one profile, which is used to denote the vector of strategies selected by the players.
Strategies of opponents of player $i$ is expressed as $s_{-i}$, and a profile can be shown as $\sigma=\{s_i, s_{-i}\}$.
$u_i(\sigma) = u_i(s_i, s_{-i})$ is the player $i$' utility.

When formulating a problem of CRN into a game, the components of two are listed in Table~\ref{game_crn_component}.

\begin{table}
\centering
\begin{tabular}{|c|c|}
\hline 
Elements of a game & Elements of one CRN \\ 
\hline 
Player $\mathcal{N}$ & secondary users \\ 
\hline 
Strategies for player $i$, $\sum_i$  & working channels, transmission power, modulation, etc. \\ 
\hline 
Utility of player $i$, $u_i$ & Performance in respect of SINR, Throughput, etc. \\ 
\hline 
\end{tabular} 
\caption{CRN and corresponding game}
\label{game_crn_component}
\end{table}

\subsubsection*{Nash Equilibrium}

A strategy profile $s\in S$ is a Nash Equilibrium (NE) if for any player $i$ and each alternate strategy $s_i'$, there is
 \[ u_i(s_i, s_{-i}) \geq u_i(s_i', s_{-i})\]
which means no unilateral deviation in strategy by any single player is profitable for that player.


There exists other equilibrium conceptions. \ie Pareto Equilibrium (PE). 
Pareto Equilibrium is a subset of NE, PE is an action profile $\bar{s}$ such that there does not exist profile $s$ with $u_i(s)\leq u_i(\bar{s})$ for each $i\in N$, and meanwhile $u_i(s)< u_i(\bar{s})$ for at least one $i\in N$.
PE is the necessary condition of the global optimality and accordingly is more favoured, but its application in communication system is much less than NE because it is not easy to obtain PE.

Nash equilibrium is a conceptual tool and a prediction about the rational strategic behaviour by agents in situations of conflict, hence, it carries great importance to know how much computational effort needed to compute NE.
%Unlucky, the computation of NE is usually a combinatorial optimization problem (chapter 2)~\cite{}.
% but in some special cases, 

%\subsubsection*{Different Games in Nutshell}



\subsubsection*{Potential Game}

A potential game is a tuple $\lambda=(\mathcal{N},(\sum_i)_{i \in \mathcal{N}},(u_i)_{i\in \mathcal{N}})$, which satisfies the following, if there exists a function $\phi: \sum\rightarrow \mathbb{R}$, such that for every $i\in \mathcal{N}$, for every $s_{-i}\in \sum_{-i}$, and every $s_i, s_i'\in \sum_i$:
 \[ u_i(s_i, s_{-i})-u_i(s_i', s_{-i}) = \phi_i(s_i, s_{-i})-\phi_i(s_i', s_{-i})\]
It is easy to see that the design of function $\phi$ is the key point to form a potential game.



\subsection{Congestion Game}
Congestion game is a special type of potential game, which has extra conditions, but also yield favourable characteristics.
Congestion game is an attractive game model which describes one kind of problem where nodes compete for limited resources in a non-cooperative manner~\cite{Voecking06congestiongames}.
%In congestion game, player pays for the resources it occupies.
%Particularly, the payment for one resource is monotonically increasing with the \textit{number} of players occupying that resource, and each player tries to minimize its payment~\footnote{Another way to describe congestion game: player gets benefit by using a certain resource, the benefit is monotonically decreasing with the number of players on that resource, each player tries to maximize its welfare}. 
This game can be used to formulate many problems in realistic world, \eg minimisation of commuting time on the road for commuters, minimization of energy consumption in mobile cloud computing system~\cite{game_cloudcomputing_energy12}.

Before giving the definition, we introduce an exemplary congestion game called \textit{server matching}.
Consider a couple of self-interested clients and servers.
Each client should access one server.
The latency of one server increases with the \textit{number of clients} attached to it.
If these clients greedily choose a permissible server because of smaller predicted latency, then after finite number of switches, no client has motivation to switch any more.
%
More formally, a congestion game is composed of players (the self-interested clients) and resources (servers), where players are allowed to choose certain resources to use. There is cost (latency) generated on a resource for the player whenever the player uses that resource, and the cost is monotonic increasing with the number of players using it. 
If every player greedily searches the allowed resources to decrease its cost, the dynamics will cease a stable state called \emph{Nash Equilibrium} (NE), where no player has motivation to adopt a new set of resources unilaterally. 
An important aspect of this convergence is a value called \textit{potential} which monotonically decreases with the update of players in the convergence process.

Now we give the formal definition of congestion game.


A congestion game \cite{Rosenthal}\cite{Voecking06congestiongames} can be expressed by a tuple $\lambda=(\mathcal{N},\mathcal{R},(\sum_i)_{i \in \mathcal{N}},(g_r)_{r\in \mathcal{R}})$, where $\mathcal{N}=\left\{1,\ldots,N\right\}$ denotes the set of players (each each is labeled with a unique index number), $\mathcal{R}=\left\{1,\ldots,m\right\}$ the set of resources, $\Sigma_{i\in\mathcal{N}} \subseteq 2^{\mathcal{R}}$ is the strategy space of player $i$. 
Under strategy profile $\sigma=(\sigma_1,\sigma_2,\cdots \sigma_N)$, player $i$ chooses strategy $\sigma_i\in \Sigma_i$, and the total number of users using resource $r$ is $n_r(\sigma)=|\{i\mid r\in \sigma_i\}|$. 
The cost $g_r: \mathbb{N}\rightarrow \mathbb{Z}$ is a function of the number of users for resource $r$, $g_r^i=\sum_{r\in \sigma_i} g_r(n_r(\sigma))$. 
In our paper, $g_r^i$ is referred as \textit{congestion} and is Monotonic.

Rosenthal's potential function $\phi:\sigma_1\times\sigma_2\times\cdots\times\sigma_n\rightarrow Z$ is defined as:
\begin{equation}
\label{4}
\begin{split}
G(\sigma) 
& =\sum\limits^{}_{r\in \mathcal{R}} \sum\limits^{n_r(\sigma)}_{i=1} g_r(i)\\
& =\sum\limits_{i\in \mathcal{N}} \sum\limits^{}_{r\in \sigma_i} g_r(n_r^i(\sigma))\\
\end{split}
\end{equation}
$n_r^i(\sigma)$ means the number of players using resource $r$ and \textit{their indices are smaller than or equal to $i$}. 
Note that the potential is \textit{not} the sum of congestions experienced by every user. 
The change of the potential caused by one player's unilateral move from $\sigma$ to $\sigma'$ is equivalent to the change of gain (or loss) of that player.
\begin{equation}
\label{5}
\varDelta G(\sigma_i \rightarrow \sigma_i') = g^i(\sigma_i',\sigma_{-i}) - g^i(\sigma_i,\sigma_{-i})
\end{equation}
$\sigma_{-i}$ is the strategy profile for all players except for $i$.
As every congestion game is a potential game, and the total potential is finite, thus the number of improvements is upper-bounded by $2\cdot\sum\limits^{}_{r\in \mathcal{R}} \sum\limits^{n_r(\sigma)}_{i=1} g_r(i)$ \cite{Voecking06congestiongames}.

%\section{Application of congestion game in the design of decentralised algorithm}
%\todo[inline]{expand: the application of potential game in CRN}



\section{Optimization}
As discussed in Chapter~\ref{INTRODUCTION}, the available radio resources such as spectrum are very limited, at the meantime, new services raise great requirements for these resources.
Resource allocation and its optimization are the methods to accommodate the needs.
Various optimization problems are formulated to improve radio resource usage in CRN~\cite{cacao_ca_2011, fuzzy_decision_09, resourceAllocation_imperfectSensing_2012}.
Optimization can be conducted from either global view or individual perspective.

Many wireless resource allocation problems are formulated as constrained optimization problems.
Table~\ref{opt_table} shows the commonly used parameters, objectives and constraints in optimization problems of wireless communication.
Part of the contents in Table~\ref{opt_table} refers \cite{Han:2008:RAW:1457343}.

\begin{table}
\begin{tabular}{|C{2.5cm}|C{3.5cm} | C{3.5cm} | C{3.5cm}|}
\hline 
 & Parametres & Optimization goals & Constraints \\ 
\hline 
Application layer & source-coding rate, buffer priority, packet arrival rate & minimal delay & base layer transmission, strict delay requirement \\ 
\hline 
Network layer & routing path & end to end delay/throughput & maximal hops, security concerns \\ 
\hline 
MAC layer & transmission frequency, transmission priorities & maximal overall throughput, minimal buffer overflow probability & contentions, time/frequency slot \\ 
\hline
Physical layer & transmission power, modulation, channel coding rate & minimal over power consumption, maximal throughput, minimal BER & maximal transmission power, caused interference on licensed users, available channel coding rate \\ 
\hline
\end{tabular} 
\caption{Optimization problem of cognitive radio networks}
\label{opt_table} 
\end{table}

The solution to an optimization problems is decided by its property, \ie convex, linear, integer, or non-convex non-linear, etc.


%Optimization outputs the results with the global information.
%Optimization is implemented on one entity, thus it is naturally suitable in centralized scheme.
